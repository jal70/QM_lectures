\hypertarget{Objetivos}{\section{\texorpdfstring{Objetivos\protect\hyperlink{Objetivos}{¶}}{Objetivos¶}}\label{Objetivos}}

En esta sección estudiaremos los conceptos de

\begin{itemize}
\tightlist
\item
  Espacio vectorial
\item
  Isomorfismo de espacios vectoriales
\item
  Espacio dual
\item
  Subespacio vectorial
\item
  Suma directa de subespacios
\item
  Combinación lineal, dependencia e independencia lineal
\item
  Bases y dimensión
\item
  Producto tensorial
\item
  Espacio de Hilbert
\item
  Producto escalar
\item
  Notación de Dirac
\end{itemize}

Nota, nota nota

\begin{itemize}
\tightlist
\item
  uno
\item
  dos
\end{itemize}

Al finalizar la sección los estudiantes deben comprender los conceptos
básicos, deben ser capaces de verificar las características que definen
un espacio vectorial, verificar la independencia lineal de un conjunto
de vectores, expresar un vector como CL, de la base, hallar las
componentes de un vector proyectando sobre los elementos de una base
ortonormal.

Esta sección es sobre algunos conceptos matemáticos indispensables para
la descripción de la Mecánica Cuántica.

\hypertarget{Espacio-Vectorial-o-Espacio-Lineal}{\subsection{\texorpdfstring{Espacio
Vectorial o Espacio
Lineal\protect\hyperlink{Espacio-Vectorial-o-Espacio-Lineal}{¶}}{Espacio Vectorial o Espacio Lineal¶}}\label{Espacio-Vectorial-o-Espacio-Lineal}}

Es un conjunto de objetos (\$\textbackslash{}psi\$,
\$\textbackslash{}phi\$, \$\textbackslash{}varphi\$, etc) que llamaremos
vectores, asociados a un campo numérico (\$\textbackslash{}mathbb\{R\}\$
o \$\textbackslash{}mathbb\{C\}\$) que llamaremos escalares.

\begin{itemize}
\tightlist
\item
  \$\textbackslash{}psi, \textbackslash{}phi, \textbackslash{}varphi
  \textbackslash{}dots \textbackslash{}in V\$
\item
  \$a, b, c \textbackslash{}dots \textbackslash{}in
  \textbackslash{}mathbb\{R\},\textbackslash{}mathbb\{C\}\$
\end{itemize}

Este conjunto tiene dos operaciones asociadas

\begin{itemize}
\tightlist
\item
  Suma de vectores
\item
  Producto por escalares
\end{itemize}

\hypertarget{Propiedades-de-la-suma-de-vectores}{\subsection{\texorpdfstring{Propiedades
de la suma de
vectores\protect\hyperlink{Propiedades-de-la-suma-de-vectores}{¶}}{Propiedades de la suma de vectores¶}}\label{Propiedades-de-la-suma-de-vectores}}

La suma es una típica operación abeliana (conmutativa)

\begin{itemize}
\tightlist
\item
  Clausura: \$\textbackslash{}phi+\textbackslash{}varphi
  \textbackslash{}in V\$

  \begin{itemize}
  \tightlist
  \item
    La suma nunca se escapa del espacio
  \end{itemize}
\item
  Conmutatividad: \$\textbackslash{}phi+\textbackslash{}varphi =
  \textbackslash{}varphi+\textbackslash{}phi\$
\item
  Asociatividad:
  \$(\textbackslash{}phi+\textbackslash{}varphi)+\textbackslash{}psi =
  \textbackslash{}varphi+(\textbackslash{}phi+\textbackslash{}psi)=
  \textbackslash{}varphi+\textbackslash{}phi+\textbackslash{}psi\$

  \begin{itemize}
  \tightlist
  \item
    El orden es indiferente: no hay que declararlo
  \end{itemize}
\item
  Existencia del elemento neutro para todos los elementos:
  \$O+\textbackslash{}phi=\textbackslash{}phi+O=\textbackslash{}phi\$
\item
  Existencia del vector simétrico, opuesto o inverso de cada elemento:
  \$\textbackslash{}phi+(-\textbackslash{}phi)=O\$
\end{itemize}

\hypertarget{Propiedades-de-la-multiplicaciuxf3n-por-escalares}{\subsection{\texorpdfstring{Propiedades
de la multiplicación por
escalares\protect\hyperlink{Propiedades-de-la-multiplicaciuxf3n-por-escalares}{¶}}{Propiedades de la multiplicación por escalares¶}}\label{Propiedades-de-la-multiplicaciuxf3n-por-escalares}}

\begin{itemize}
\tightlist
\item
  Clausura: \$a\textbackslash{}phi \textbackslash{}in V\$
\item
  Distributibidad:

  \begin{itemize}
  \tightlist
  \item
    En suma de vectores
    \$a(\textbackslash{}phi+\textbackslash{}varphi)=a\textbackslash{}phi+a\textbackslash{}varphi\$
  \item
    En suma de escalares
    \$(a+b)\textbackslash{}phi=a\textbackslash{}phi+b\textbackslash{}phi\$
  \end{itemize}
\item
  Asociatividad:
  \$(ab)\textbackslash{}phi=a(b\textbackslash{}phi)=ab\textbackslash{}phi\$
\item
  Existencia del elemento neutro para todos los vectores:
  \$1\textbackslash{}phi = \textbackslash{}phi\$
\end{itemize}

\hypertarget{Subespacio-vectorial}{\subsection{\texorpdfstring{Subespacio
vectorial\protect\hyperlink{Subespacio-vectorial}{¶}}{Subespacio vectorial¶}}\label{Subespacio-vectorial}}

Es un subconjunto de elementos de un espacio vectorial
(\$\textbackslash{}psi\$, \$\textbackslash{}phi\$,
\$\textbackslash{}varphi\$, etc)

\begin{itemize}
\tightlist
\item
  Satisface las propiedades de las operaciones por sí mismo.
\item
  Atención a las propiedades de clausura y existencia de vector
  simétrico
\end{itemize}

\hypertarget{Combinaciuxf3n-lineal}{\subsection{\texorpdfstring{Combinación
lineal\protect\hyperlink{Combinaciuxf3n-lineal}{¶}}{Combinación lineal¶}}\label{Combinaciuxf3n-lineal}}

Es una operación donde se conmbinan la suma de vectores y el producto
por escalares

\$\$\textbackslash{}psi=\textbackslash{}sum\_\{i=1\}\^{}\{n\}a\_i\textbackslash{}phi\_i\$\$

El vector \$\textbackslash{}psi\$ es la combinación lineal \textbf{(CL)}
de los \$n\$ vectores
\$\textbackslash{}\{\textbackslash{}phi\_1\textbackslash{}dots\textbackslash{}phi\_n\textbackslash{}\}\$
con \textbf{coeficientes} escalares \$\textbackslash{}\{a\_1
\textbackslash{}dots a\_n\textbackslash{}\}\$

\hypertarget{Dependencia-lineal}{\subsection{\texorpdfstring{Dependencia
lineal\protect\hyperlink{Dependencia-lineal}{¶}}{Dependencia lineal¶}}\label{Dependencia-lineal}}

Un conjunto de vectores
\$\textbackslash{}\{\textbackslash{}phi\_n\textbackslash{}\}\$ es
linealmente independiente \textbf{(LI)} si cualquier \textbf{CL} de la
siguiente forma:
\$\$O=\textbackslash{}sum\_\{i=1\}\^{}\{n\}a\_i\textbackslash{}phi\_i,\$\$
solo es posible cuando todos los coeficientes son nulos
\$\$a\_1=0,\textbackslash{}dots a\_n=0\$\$.

En caso contrario el conjunto es linealmente dependiente \textbf{(LD)}

\hypertarget{Dependencia-lineal}{\subsection{\texorpdfstring{Dependencia
lineal\protect\hyperlink{Dependencia-lineal}{¶}}{Dependencia lineal¶}}\label{Dependencia-lineal}}

Cuando un conjunto es LD siempre se puede despejar uno de los vectores
con coeficiente no-nulo en función de los demás. Si
\$a\_k\textbackslash{}neq 0\$:
\$\$\textbackslash{}phi\_k=-(\textbackslash{}sum\_\{i=1\}\^{}\{k-1\}\textbackslash{}frac\{a\_i\}\{a\_k\}\textbackslash{}phi\_i
+
\textbackslash{}sum\_\{i=k+1\}\^{}\{n\}\textbackslash{}frac\{a\_i\}\{a\_k\}\textbackslash{}phi\_i)\$\$

\subsection{\texorpdfstring{Dimensión
(\$dim{[}V{]}\$)\protect\hyperlink{Dimensiuxf3n-ux28ux24dimux5cux255BVux5cux255Dux24ux29}{¶}}{Dimensión (\$dim{[}V{]}\$)¶}}\label{Dimensiuxf3n-ux28ux24dimux7bux5bux7dVux7bux5dux7dux24ux29}

\begin{itemize}
\tightlist
\item
  Es el número máximo de vectores que pueden formar parte de un conjunto
  LI en un espacio vectorial.
\item
  En caso que sea posible agregar vectores indefinidamente a un conjunto
  LI, el espacio tiene dimensión infinita.
\end{itemize}

\hypertarget{Base}{\subsection{\texorpdfstring{Base\protect\hyperlink{Base}{¶}}{Base¶}}\label{Base}}

\begin{itemize}
\tightlist
\item
  Es cualquier conjunto de \$dim{[}V{]}=N\$ vectores LI.
\item
  Si se agrega un vector adicional \$\textbackslash{}psi\$, el conjunto
  es LD
  \$\$-a\_0\textbackslash{}psi+\textbackslash{}sum\_\{i=1\}\^{}\{N\}a\_i\textbackslash{}phi\_i=O\$\$
  es posible para algún \$a\_0\textbackslash{}neq 0\$. Por lo tanto
  \$\$\textbackslash{}psi=\textbackslash{}frac\{1\}\{a\_0\}\textbackslash{}sum\_\{i=1\}\^{}\{N\}a\_i\textbackslash{}phi\_i\$\$
\end{itemize}

\hypertarget{Base}{\subsection{\texorpdfstring{Base\protect\hyperlink{Base}{¶}}{Base¶}}\label{Base}}

\begin{itemize}
\item
  Es un conjunto de vectores que pueden expandir a todo el espacio.
\item
  Todo vector del espacio se puede escribir como CL de los vectores de
  la base.
\end{itemize}

\hypertarget{Suma-directa-de-subespacios}{\subsection{\texorpdfstring{Suma
directa de
subespacios\protect\hyperlink{Suma-directa-de-subespacios}{¶}}{Suma directa de subespacios¶}}\label{Suma-directa-de-subespacios}}

Se dice que \$V\$ es la suma directa de dos subespacios \$S\$ y \$T\$ si
cualquier vector de V se escribe como una única suma de un elemento de
cada subespacio.

\$\$\textbackslash{}psi=\textbackslash{}phi+\textbackslash{}varphi,\$\$

donde \$\textbackslash{}psi\textbackslash{}in V\$,
\$\textbackslash{}phi\textbackslash{}in S\$ y
\$\textbackslash{}varphi\textbackslash{}in T\$

Se escibe:

\$\$V=S\textbackslash{}oplus T\$\$

Las dimensiones se suman

\$\$dim{[}V{]}=dim{[}S{]}+dim{[}T{]}\$\$

Un ejemplo es el espacio vectorial \$\textbackslash{}mathbb\{R\}\^{}3 =
P \textbackslash{}oplus L\$, donde \$P\$ es el conjunto de puntos de un
plano y \$L\$ es una recta que no está contenida en él.

\hypertarget{Suma-directa-de-subespacios}{\subsection{\texorpdfstring{Suma
directa de
subespacios\protect\hyperlink{Suma-directa-de-subespacios}{¶}}{Suma directa de subespacios¶}}\label{Suma-directa-de-subespacios}}

En MC encontraremos sistemas donde es conveniente representar el espacio
de Hilbert de un sistema como la suma directa de subespacios con ciertas
propiedades.

En el contexto de la Mecánica Cuántica, la suma directa tiene relevancia
en la descripción de la separación del espacio de estados en sectores
que se diferencian y se separan por el valor de alguna magnitud física
observable. Cuando estudiemos la suma de momento angular veremos un
ejemplo de muchísimo interés.

\hypertarget{Spoiler-alert}{\subsubsection{\texorpdfstring{Spoiler
alert\protect\hyperlink{Spoiler-alert}{¶}}{Spoiler alert¶}}\label{Spoiler-alert}}

\begin{itemize}
\tightlist
\item
  ¿Alguien ha dicho espacio de Hilbert?
\item
  ¿Qué es eso?
\end{itemize}

\hypertarget{Espacio-Dual}{\subsection{\texorpdfstring{Espacio
Dual\protect\hyperlink{Espacio-Dual}{¶}}{Espacio Dual¶}}\label{Espacio-Dual}}

Se denomina espacio dual \$V\^{}\{*\}\$ a \$V\$, al conjunto de todas
las funciones lineales que toman un vector y producen un escalar.

\$\$f(\textbackslash{}sum\_\{i=1\}\^{}\{n\}a\_i\textbackslash{}psi\_i)=\textbackslash{}sum\_\{i=1\}\^{}\{n\}a\_if(\textbackslash{}psi\_i)\$\$

\begin{itemize}
\tightlist
\item
  \$V\^{}\{*\}\$ es un espacio vectorial \textbf{(Ejercicio)}

  \begin{itemize}
  \tightlist
  \item
    La función lineal nula es cumple \$f\_O(\textbackslash{}psi)=0\$
    para cualquier \$\textbackslash{}psi\$
  \end{itemize}
\item
  Si \$dim{[}V{]}\textless{}\textbackslash{}infty\$, entonces
  \$dim{[}V{]}=dim{[}V\^{}\{*\}{]}\$
\end{itemize}

\subsection{\texorpdfstring{Base de
\$V\^{}\{*\}\$\protect\hyperlink{Base-de-ux24Vux5cux255Eux5cux257Bux2aux5cux257Dux24}{¶}}{Base de \$V\^{}\{*\}\$¶}}\label{Base-de-ux24Vux5cux5eux7bux7dux5cux7bux2aux5cux7dux24}

Tomemos una base \$\textbackslash{}\{\textbackslash{}phi\_i;
i=1\textbackslash{}dots N\textbackslash{}\}\$ y consideremos las
siguientes \$N\$ funciones lineales:

\$\$f\_i(\textbackslash{}phi\_j)=\textbackslash{}delta\_\{ij\}\$\$

\begin{itemize}
\tightlist
\item
  Las funciones \$\textbackslash{}\{f\_i; i=1\textbackslash{}dots
  N\textbackslash{}\}\$ son LI \textbf{(Ejercicio)}
\end{itemize}

\begin{itemize}
\tightlist
\item
  Cualquier función lineal \$g\$, no nula, cumple:
  \$\$g()-\textbackslash{}sum\_\{i=1\}\^{}\{n\}\textbackslash{}alpha\_if()
  = f\_O(),\$\$ con algunos coeficientes \$\textbackslash{}alpha\_i\$ no
  nulos \textbf{(Ejercicio)}. Es decir,
\end{itemize}

\$\$g()=\textbackslash{}sum\_\{i=1\}\^{}\{n\}\textbackslash{}alpha\_if()\$\$

\$g\$ es CL de las funciones \$\textbackslash{}\{f\_i;
i=1\textbackslash{}dots N\textbackslash{}\}\$

\hypertarget{Producto-tensorial}{\subsection{\texorpdfstring{Producto
tensorial\protect\hyperlink{Producto-tensorial}{¶}}{Producto tensorial¶}}\label{Producto-tensorial}}

Si tenemos dos espacios vectoriales \$V\$ y \$W\$, sobre el mismo campo
de escalares, podemos considerar el conjunto de todas las funciones
bilineales \$B(V,W)\$ que toman vectores de ambos espacios y regresan
escalares.

\$\$f(\textbackslash{}psi,\textbackslash{}phi)
\textbackslash{}longrightarrow \textbackslash{}mathbb\{C\}\$\$

Una forma muy intuitiva de entender el concepto de producto tensorial es
a través del estudio de las funciones escalares lineales de dos
argumentos vectoriales.

En este caso consideramos una función que toma dos vectores y produce un
número complejo.

La función es lineal en los dos argumentos

\begin{itemize}
\tightlist
\item
  Lineal en el primer argumento (vector de \$V\$)
\end{itemize}

\$\$f(a\_1\textbackslash{}psi\_1+a\_2\textbackslash{}psi\_2,\textbackslash{}phi)
= a\_1f(\textbackslash{}psi\_1,\textbackslash{}phi) +
a\_2f(\textbackslash{}psi\_2,\textbackslash{}phi)\$\$

\begin{itemize}
\tightlist
\item
  Lineal en el segundo argumento (vector de \$W\$)
\end{itemize}

\$\$f(\textbackslash{}psi,b\_1\textbackslash{}phi\_1+b\_2\textbackslash{}phi\_2)
= b\_1f(\textbackslash{}psi,\textbackslash{}phi\_1) +
b\_2f(\textbackslash{}psi,\textbackslash{}phi\_2)\$\$

\hypertarget{Base-del-espacio-ux24Bux28Vux2cWux29ux24}{\subsection{\texorpdfstring{Base
del espacio
\$B(V,W)\$\protect\hyperlink{Base-del-espacio-ux24Bux28Vux2cWux29ux24}{¶}}{Base del espacio \$B(V,W)\$¶}}\label{Base-del-espacio-ux24Bux28Vux2cWux29ux24}}

\begin{itemize}
\tightlist
\item
  Se puede probar que \$B(V,W)\$ es un espacio vectorial, tiene
  dimensión \$dim{[}V{]}dim{[}W{]}\$ y construir una base
\end{itemize}

Podemos construir una base para el espacio \$B(V,W)\$ a partir de las
bases de los espacios duales \$V\^{}\{*\}\$ (\$f\_i\$) y \$W\^{}\{*\}\$
(\$g\_i\$). Al fin y al cabo se trata de funciones.

Las funciones \$F\_\{ij\}\textbackslash{}equiv f\_ig\_j\$ cumplen con la
bilinealidad. Además,

\$\$F\_\{ij\}(\textbackslash{}phi\_k\^{}\{V\},\textbackslash{}phi\_l\^{}\{W\})=\textbackslash{}delta\_\{ik\}\textbackslash{}delta\_\{jl\}\$\$

\begin{itemize}
\tightlist
\item
  Las funciones \$F\_\{ij\}\$ son una base de \$B(V,W)\$
\end{itemize}

\begin{itemize}
\item
  Ventaja de usar el espacio de funciones: se ve fácilmente que la
  dimensión es el producto de las dimensiones de los espacios duales de
  los espacios V y W
\item
  Solo hay que multiplicar funciones escalares lineales de cada espacio
  dual y obtenemos las funciones del espacio B
\end{itemize}

\hypertarget{Definiciones}{\subsubsection{\texorpdfstring{Definiciones\protect\hyperlink{Definiciones}{¶}}{Definiciones¶}}\label{Definiciones}}

\begin{itemize}
\tightlist
\item
  \$B(V,W)\textbackslash{}equiv V\^{}\{*\}\textbackslash{}otimes
  W\^{}\{*\}\$ es el prodcuto tensorial de los espacios \$V\^{}\{*\}\$ y
  \$W\^{}\{*\}\$
\item
  \$V\textbackslash{}otimes W \textbackslash{}equiv
  (V\^{}\{*\}\textbackslash{}otimes W\^{}\{*\})\^{}\{*\}\$ es el
  prodcuto tensorial de los espacios \$V\$ y \$W\$
\end{itemize}

\begin{itemize}
\tightlist
\item
  Decimos que B es el producto tensorial de los duales V \emph{y W}
\item
  La definición que estamos buscando es la de producto tensorial de los
  espacios V y W
\item
  La base del producto V x W es el producto de los vectores de las bases
  de V y W
\end{itemize}

\hypertarget{Producto-tensorial}{\subsection{\texorpdfstring{Producto
tensorial\protect\hyperlink{Producto-tensorial}{¶}}{Producto tensorial¶}}\label{Producto-tensorial}}

En MC el concepto de producto tensorial es utilizado para construir el
espacio de Hilbert de un sistema compuesto de subsistemas más simples.

\begin{itemize}
\tightlist
\item
  Ejemplo: Espacio de Hibert de un sistema de dos partículas libres es
  el producto tensorial de los espacios de Hilbert de cada partícula
\end{itemize}

\hypertarget{Spoiler-alert}{\subsubsection{\texorpdfstring{Spoiler
alert\protect\hyperlink{Spoiler-alert}{¶}}{Spoiler alert¶}}\label{Spoiler-alert}}

\begin{itemize}
\tightlist
\item
  ¿Alguien ha dicho espacio de Hilbert de nuevo?
\item
  ¿Qué es eso?
\end{itemize}

\hypertarget{Producto-escalar-y-Espacio-de-Hilbert}{\subsection{\texorpdfstring{Producto
escalar y Espacio de
Hilbert\protect\hyperlink{Producto-escalar-y-Espacio-de-Hilbert}{¶}}{Producto escalar y Espacio de Hilbert¶}}\label{Producto-escalar-y-Espacio-de-Hilbert}}

Un espacio de Hilbert (\$\textbackslash{}mathcal\{H\}\$) es un espacio
vectorial que posee las siguientes características adicionales:

\begin{itemize}
\tightlist
\item
  Tiene un producto escalar \textbf{(PE)} estrictamente positivo

  \begin{itemize}
  \tightlist
  \item
    \$(\textbackslash{}phi,\textbackslash{}psi)
    \textbackslash{}longrightarrow \textbackslash{}mathbb\{C\}\$
  \item
    Estrictamente positivo:

    \begin{itemize}
    \tightlist
    \item
      \$(\textbackslash{}psi,\textbackslash{}psi) \textbackslash{}equiv
      \textbar{}\textbar{}\textbackslash{}psi\textbar{}\textbar{}\^{}\{2\}\textbackslash{}geq
      0\$
    \item
      \$\textbar{}\textbar{}\textbackslash{}psi\textbar{}\textbar{}\^{}\{2\}
      = 0 \textbackslash{}Leftrightarrow \textbackslash{}psi=O\$
    \end{itemize}
  \item
    Se define una función de distancia a partir del PE:
    \$d(\textbackslash{}phi,\textbackslash{}psi) \textbackslash{}equiv
    \textbar{}\textbar{}(\textbackslash{}phi-\textbackslash{}psi)\textbar{}\textbar{}\$
  \end{itemize}
\end{itemize}

\begin{itemize}
\tightlist
\item
  Es separable

  \begin{itemize}
  \tightlist
  \item
    Se puede conseguir una sucesión de Cauchy en
    \$\textbackslash{}mathcal\{H\}\$ que tiene elementos arbitrariamente
    cerca de cualquier punto del espacio.

    \begin{itemize}
    \tightlist
    \item
      Para todo \$\textbackslash{}psi \textbackslash{}in
      \textbackslash{}mathcal\{H\}\$ y
      \$\textbackslash{}epsilon\textgreater{}0\$ hay un \$n\$ tal que el
      elemento \$\textbackslash{}psi\_n\$ de la sucesión cumple
      \$d(\textbackslash{}psi\_n,\textbackslash{}psi)\textless{}\textbackslash{}epsilon\$
    \end{itemize}
  \end{itemize}
\end{itemize}

\begin{itemize}
\tightlist
\item
  Es completo

  \begin{itemize}
  \tightlist
  \item
    Toda sucesión de Cauchy converge a un elemento del espacio
    \$\textbackslash{}mathcal\{H\}\$
  \end{itemize}
\end{itemize}

\hypertarget{Otras-propiedades-del-PE}{\subsubsection{\texorpdfstring{Otras
propiedades del
PE\protect\hyperlink{Otras-propiedades-del-PE}{¶}}{Otras propiedades del PE¶}}\label{Otras-propiedades-del-PE}}

\begin{itemize}
\tightlist
\item
  \$(\textbackslash{}phi,\textbackslash{}psi) =
  (\textbackslash{}psi,\textbackslash{}phi)\^{}\{*\}\$
\item
  Es lineal en el segundo argumento y antilineal en el primer argumento

  \begin{itemize}
  \tightlist
  \item
    \$(a\_1\textbackslash{}psi\_1+a\_2\textbackslash{}psi\_2,\textbackslash{}phi)
    = a\_1\^{}\{*\} (\textbackslash{}psi\_1,\textbackslash{}phi) +
    a\_2\^{}\{*\} (\textbackslash{}psi\_2,\textbackslash{}phi)\$
  \item
    \$(\textbackslash{}psi,b\_1\textbackslash{}phi\_1+b\_2\textbackslash{}phi\_2)
    = b\_1 (\textbackslash{}psi,\textbackslash{}phi\_1) + b\_2
    (\textbackslash{}psi,\textbackslash{}phi\_2)\$
  \end{itemize}
\item
  Cumple la desigualdad de Schwarz

  \begin{itemize}
  \tightlist
  \item
    \$\textbar{}(\textbackslash{}phi,\textbackslash{}psi)\textbar{}\^{}\{2\}
    \textbackslash{}leq
    (\textbackslash{}phi,\textbackslash{}phi)(\textbackslash{}psi,\textbackslash{}psi)\$
  \end{itemize}
\item
  La distancia cumple la desigualdad triangular

  \begin{itemize}
  \tightlist
  \item
    \$d(\textbackslash{}phi,\textbackslash{}psi) \textbackslash{}leq
    d(\textbackslash{}phi,\textbackslash{}psi') +
    d(\textbackslash{}psi',\textbackslash{}psi)\$
  \end{itemize}
\end{itemize}

Recordar: el producto escalar
\$(\textbackslash{}phi,\textbackslash{}psi)\$ representa la proyección
del vector \$\textbackslash{}phi\$ sobre el vector
\$\textbackslash{}psi\$.

\includegraphics[width=1.87500in]{imagenes/Dot_Product.png}

De No machine-readable author provided. \textless{}a
href=``//commons.wikimedia.org/w/index.php?title=User:Mazin07\&amp;action=edit\&amp;redlink=1''
class=``new'' title=``User:Mazin07 (page does not
exist)''\textgreater{}Mazin07\textless{}/a\textgreater{} assumed (based
on copyright claims). - No machine-readable source provided. Own work
assumed (based on copyright claims)., Dominio público,
\href{https://commons.wikimedia.org/w/index.php?curid=3899178}{Enlace}

\hypertarget{Base-ortogonal}{\subsection{\texorpdfstring{Base
ortogonal\protect\hyperlink{Base-ortogonal}{¶}}{Base ortogonal¶}}\label{Base-ortogonal}}

Recordemos ortogonalidad \$\$(\textbackslash{}phi,\textbackslash{}psi)=0
\textbackslash{}Leftrightarrow
\textbackslash{}phi\textbackslash{}perp\textbackslash{}psi\$\$

\begin{itemize}
\tightlist
\item
  Siempre es posible construir una base de vectores ortogonales para un
  espacio \$\textbackslash{}mathcal\{H\}\$
\item
  Si además se escogen vectores unitarios, la base es llamada ortonormal
  \$\$\textbackslash{}\{\textbackslash{}phi\_i, i=1\textbackslash{}dots
  dim{[}\textbackslash{}mathcal\{H\}{]} /
  (\textbackslash{}phi\_i,\textbackslash{}phi\_j)=\textbackslash{}delta\_\{ij\}
  \textbackslash{}\}\$\$
\end{itemize}

\hypertarget{PE-y-espacio-dual}{\subsection{\texorpdfstring{PE y espacio
dual\protect\hyperlink{PE-y-espacio-dual}{¶}}{PE y espacio dual¶}}\label{PE-y-espacio-dual}}

El PE provee una conexión natural entre los elementos de
\$\textbackslash{}mathcal\{H\}\$ y su dual
\$\textbackslash{}mathcal\{H\}\^{}\{*\}\$

\begin{itemize}
\tightlist
\item
  Consideramos \$f\_i \textbackslash{}equiv (\textbackslash{}psi\_i,)\$
\item
  \$f\_i(\textbackslash{}phi\_j) = \textbackslash{}delta\_\{ij\}\$ (es
  un elemento del espacio dual)
\item
  En general \$f\_\{\textbackslash{}psi\} \textbackslash{}equiv
  (\textbackslash{}psi,) \textbackslash{}in
  \textbackslash{}mathcal\{H\}\^{}\{*\}\$
\end{itemize}

\hypertarget{Ejemplos-de-espacios-de-Hlibert}{\subsection{\texorpdfstring{Ejemplos
de espacios de
Hlibert\protect\hyperlink{Ejemplos-de-espacios-de-Hlibert}{¶}}{Ejemplos de espacios de Hlibert¶}}\label{Ejemplos-de-espacios-de-Hlibert}}

\begin{itemize}
\tightlist
\item
  \$\textbackslash{}mathbb\{R\}\^{}\{n\}\$ es el conjunto de las
  n-tuplas de números reales
\item
  El conjunto de los desplazamientos en el sentido que se usa en la
  Mecánica Clásica
  \$\$(\textbackslash{}phi,\textbackslash{}psi)=\textbackslash{}sum\_\{i=1\}\^{}N\textbackslash{}phi\_i\textbackslash{}psi\_i\$\$
\item
  Espacio de las funciones
  \$\textbackslash{}psi(x):\textbackslash{}mathbb\{R\}
  \textbackslash{}rightarrow \textbackslash{}mathbb\{C\}\$ de cuadrado
  integrable \$\$\textbackslash{}int
  \textbar{}\textbar{}\textbackslash{}phi\textbar{}\textbar{}\^{}\{2\}
  dx \textbackslash{}le \textbackslash{}infty \$\$
  \$\$(\textbackslash{}phi,\textbackslash{}psi)=\textbackslash{}int
  \textbackslash{}phi\^{}\{*\}(x) \textbackslash{}psi(x) dx\$\$
\end{itemize}

\hypertarget{Notaciuxf3n-de-Dirac}{\subsection{\texorpdfstring{Notación
de
Dirac\protect\hyperlink{Notaciuxf3n-de-Dirac}{¶}}{Notación de Dirac¶}}\label{Notaciuxf3n-de-Dirac}}

\begin{itemize}
\tightlist
\item
  Llamamos \emph{ket}, \$\textbar{}\textbackslash{}psi\textgreater{}\$,
  a los elementos de \$\textbackslash{}mathcal\{H\}\$
\item
  Llamamos \emph{bra}, \$\textless{}\textbackslash{}psi\textbar{}\$, a
  los elementos de \$\textbackslash{}mathcal\{H\}\^{}\{*\}\$
\end{itemize}

Identificamos \$\$f\_\{\textbackslash{}psi\} = (\textbackslash{}psi,)
\textbackslash{}Longrightarrow
\textless{}\textbackslash{}psi\textbar{}\$\$

\$\$(\textbackslash{}psi,\textbackslash{}phi)
\textbackslash{}Longrightarrow
\textless{}\textbackslash{}psi\textbar{}\textbackslash{}phi\textgreater{}
\$\$

es el \emph{bra} \emph{ket} = \emph{braket}

\hypertarget{Propiedades}{\subsection{\texorpdfstring{Propiedades\protect\hyperlink{Propiedades}{¶}}{Propiedades¶}}\label{Propiedades}}

\$\$\textless{}\textbackslash{}psi\textbar{}\textbackslash{}phi\textgreater{}
=
\textless{}\textbackslash{}phi\textbar{}\textbackslash{}psi\textgreater{}\^{}\{*\}\$\$\$\$a\textbar{}\textbackslash{}psi\textgreater{}
+ b\textbar{}\textbackslash{}phi\textgreater{} =
a\^{}\{*\}\textless{}\textbackslash{}psi\textbar{} +
b\^{}\{*\}\textless{}\textbackslash{}phi\textbar{}\$\$\$\$\textbar{}a\textbackslash{}psi\textgreater{}
=
a\textbar{}\textbackslash{}psi\textgreater{}\$\$\$\$\textless{}a\textbackslash{}psi\textbar{}
= a\^{}\{*\}\textless{}\textbackslash{}psi\textbar{}\$\$

\hypertarget{Interpretaciuxf3n-del-PE-en-MC}{\subsection{\texorpdfstring{Interpretación
del PE en
MC\protect\hyperlink{Interpretaciuxf3n-del-PE-en-MC}{¶}}{Interpretación del PE en MC¶}}\label{Interpretaciuxf3n-del-PE-en-MC}}

En MC el estado de una sistema es representado por un elemento de un
\$\textbackslash{}mathcal\{H\}\$. Si el estado actual del sistema es
\$\textbar{}\textbackslash{}psi\textgreater{}\$ y
\$\textbar{}\textbackslash{}phi\textgreater{}\$ es un vector unitario
que representa un estado que caracteriza el sistema luego de una
medición, la cantidad
\$\textbar{}\textless{}\textbackslash{}phi\textbar{}\textbackslash{}psi\textgreater{}\textbar{}\^{}\{2\}\$
es la probabilidad de obtener dicho resultado después de la medición. Se
define

\$\$\textless{}\textbackslash{}phi\textbar{}\textbackslash{}psi\textgreater{}\$\$

como la amplitud \$\textbar{}\textbackslash{}psi\textgreater{}
\textbackslash{}Longrightarrow\textbar{}\textbackslash{}phi\textgreater{}\$

\hypertarget{El-problema-de-los-espacios-de-dimensiuxf3n-infinita}{\subsection{\texorpdfstring{El
problema de los espacios de dimensión
infinita\protect\hyperlink{El-problema-de-los-espacios-de-dimensiuxf3n-infinita}{¶}}{El problema de los espacios de dimensión infinita¶}}\label{El-problema-de-los-espacios-de-dimensiuxf3n-infinita}}

\begin{itemize}
\tightlist
\item
  Consideramos el espacio de las funciones continuas, reales en el
  intervalo \$(0,L)\$ con \$f(0)=f(L)=0\$
\item
  Es \$L\_2\$
\item
  Es un \$\textbackslash{}mathcal\{H\}\$ con el producto usual
\end{itemize}

\hypertarget{Base}{\subsection{\texorpdfstring{Base\protect\hyperlink{Base}{¶}}{Base¶}}\label{Base}}

Se usa la expresión en serie de Fourier (base ortogonal)

\$\$f(x)=\textbackslash{}sum\_\{i=1\}\^{}\{\textbackslash{}infty\}f\_i\textbackslash{}sin
(\textbackslash{}frac\{n\textbackslash{}pi x\}\{L\})\$\$

\hypertarget{Elemento-del-espacio-dual}{\subsection{\texorpdfstring{Elemento
del espacio
dual\protect\hyperlink{Elemento-del-espacio-dual}{¶}}{Elemento del espacio dual¶}}\label{Elemento-del-espacio-dual}}

¿Podemos construir una función lineal de este espacio en los números
reales, tal que, \$F\_\{y\}{[}f{]}=f(y)\$?

La solución parte por considerar la siguiente expansión:

\$\$ F\_y(x) = \textbackslash{}sum\_\{n=1\}\^{}\{\textbackslash{}infty\}
\textbackslash{}sin\{(\textbackslash{}frac\{\textbackslash{}pi
ny\}\{L\})\}
\textbackslash{}sin\{(\textbackslash{}frac\{\textbackslash{}pi
nx\}\{L\})\} \$\$

Por la unicidad de la serie de Fourier:

\$\$ F\_\{y\}{[}f{]} \textbackslash{}equiv
\textbackslash{}int\_\{0\}\^{}\{L\}dx F\_y(x)f(x) = f(y) \$\$

\hypertarget{Elementos-de-norma-infinita}{\subsection{\texorpdfstring{Elementos
de norma
infinita\protect\hyperlink{Elementos-de-norma-infinita}{¶}}{Elementos de norma infinita¶}}\label{Elementos-de-norma-infinita}}

Consideramos

\$\$\textless{}F\_y\textbar{}F\_y\textgreater{} =
\textbackslash{}sum\_\{n=0\}\^{}\{\textbackslash{}infty\}
\textbackslash{}sin\^{}\{2\} (\textbackslash{}frac\{n\textbackslash{}pi
y\}\{L\}) = \textbackslash{}infty\$\$

Revisar la definición de la Delta de Dirac \$F\_y(x)\textbackslash{}sim
\textbackslash{}delta(x-y)\$

In~{[}1{]}:

\begin{verbatim}
import matplotlib.pyplot as plt
import numpy as np

def F_yLn(x,y,L,n):
    output=0.0
    for i in range(n):
        output=output+np.sin(np.pi*i*y/L) * np.sin(np.pi*i*x/L)
    return output
\end{verbatim}

Definimos la sumatoria de la función \$F\_\{y\}(x)\$ hasta un orden dado
y un intevalo \$L\$. Es la función \$F\_\{yLn\}(x)\$:

In~{[}2{]}:

\begin{verbatim}
L1=1.0
y1=2*L1/5.0
x1 = np.arange(0.0, 1.0, 0.004)
\end{verbatim}

Consideramos los siguientes valores para el tamaño del intervalo
\$(L=L1=\$ 1.0 \$)\$ y punto de prueba \$y1=\$ 0.4. El rango de
graficación estará definido en \$x1\$

Graficamos la función

\$\$ F\_\{yLn\}(x)\textbar{}\_\{y=y1,L=L1,n=20\} \$\$

In~{[}3{]}:

\begin{verbatim}
plt.plot(x1, F_yLn(x1,y1,L1,20), 'k')
plt.axvline(x=y1, ymin=-1.0, ymax=1.0)
plt.show()
\end{verbatim}

\includegraphics{data:image/png;base64,iVBORw0KGgoAAAANSUhEUgAAAXIAAAD4CAYAAADxeG0DAAAAOXRFWHRTb2Z0d2FyZQBNYXRwbG90bGliIHZlcnNpb24zLjMuMiwgaHR0cHM6Ly9tYXRwbG90bGliLm9yZy8vihELAAAACXBIWXMAAAsTAAALEwEAmpwYAAAvdElEQVR4nO3deVxVdf4/8NeHfd9RRMQVF8RfKuqQaSiaX21GKzP3MptvVvbVFuc7NdVYk+lkZmMz+a1cWnSympqmzMxc0xalAFs0l1BEQVQQUVRA4L5/f8BBQJZ77zn3Xi739Xw8eAj3fs55f47Ai8/9nHM+V4kIiIjIebk5ugNERKQPg5yIyMkxyImInByDnIjIyTHIiYicnIcjikZEREinTp0cUZqcwNH8SwCALpH+Du4JUcuSnp5eICKR9R93SJB36tQJaWlpjihNTmDS67sBAO/fd72De0LUsiilsht6nFMrREROjkFOROTkGORERE6OQU5E5OQY5ERETs7sIFdKvaGUOqOU2lfrsTCl1Bal1K/V/4bapptERNQYS0bkbwEYXe+xxwFsE5E4ANuqvyYiIjsyO8hFZBeAwnoP3wLg7erP3wZwqzHdIldWXl6OU6dOoaKiwtFdIXIKeufI24pIXvXnpwC0bayhUmqWUipNKZWWn5+vsyy1VhcvXsRPP/+EQ4cOYfny5Y7uDpFTMOxkp1S9Q0Wj71IhIitEZICIDIiMvOYOUyIAwDPPPIOLxRfh5++HP//5zzh58qSju0TU4ukN8tNKqXYAUP3vGf1dIlf2+eefIzQsFAm9E1BcXIx169Y5uktELZ7eIF8PYEb15zMAfKJzf+TC8vLy8MsvvyA0JBS+vr7o2rUrvvnmG0d3i6jFs+Tyw3cB7AbQQymVo5T6PYDnAdyklPoVwMjqr4mssn37dgBAaGgIAOCGG27AN998A76vLFHTLLlqZYqItBMRTxGJEZHVInJWREaISJyIjBSR+le1EJlt27ZtCA0NRUBAAICqIM/Pz0dmZqaDe0bUsvHOTmoxUlNTMWTIEAAKQFWQA+D0ClEzGOTUIlRUVODXX39F7969ax7r1asXgoODkZqa6sCeEbV8DHJqEbKyslBeXo6ePXvWPObm5oZevXrh0KFDDuwZUcvHIKcW4eDBgwBQJ8gBoHv37jh8+LAjukTkNBjk1CJoQd6jR486j8fFxSE3NxeXL192RLeInAKDnFqEgwcPom3btggJCanzeFxcHADwyhWiJjDIqUU4ePDgNdMqQNXUCgD8+uuv9u4SkdNgkJPDiQgOHDjQYJB369YNADhPTtQEBjk53Pnz53Hu3Lma0K4tMDAQUVFRHJETNYFBTg53/PhxAEBsbGyDz8fFxTHIiZrAICeHO3HiBACgQ4cODT7fpUsXZGdn27NLRE6FQU4OpwV5YyPymJgYnDx5EpWVlfbsFpHTYJCTw504cQIeHh6Iiopq8PmYmBhUVlbi1KlTdu4ZkXNgkJPDnThxAtHR0XB3d2/w+ZiYGABATk6OPbtF5DQY5ORwx48fb3R+HGCQEzWHQU4Od+LEiUbnxwEGOVFzGOTkUCaTCTk5OU2OyMPDw+Hj48MgJ2oEg5wcKj8/H1euXGkyyJVSiImJYZATNYJBTg6l3QzUVJADYJATNYFBTg6Vl5cHAGjfvn2T7RjkRI1jkJNDnT59GgDQtm3bJtvFxMQgNzcXJpPJHt0icioMcnIo7SafNm3aNNkuOjoa5eXlKCgosEe3iJwKg5wc6tSpUwgLC4O3t3eT7bQRuzaCJ6KrDAlypdQjSqn9Sql9Sql3lVI+RuyXWr/Tp083O60CoOb2fQY50bV0B7lSqj2AuQAGiEgCAHcAk/Xul1zDqVOnGl1jpTaOyIkaZ9TUigcAX6WUBwA/ACcN2i+1cqdOnTJrRM4gJ2qc7iAXkVwALwI4DiAPwHkR2Vy/nVJqllIqTSmVlp+fr7cstRKnT582a0QeHBwMLy8vroBI1AAjplZCAdwCoDOAaAD+Sqnp9duJyAoRGSAiAyIjI/WWpVbg4sWLuHjxollBrpRC27ZtOSInaoARUysjAWSJSL6IlAP4CMBgA/ZLrZy515BrGOREDTMiyI8DSFJK+SmlFIARAA4YsF9q5bRQNmdEDjDIiRpjxBx5KoAPAWQA+Ll6nyv07pdaP22+29wgj4qKYpATNcDDiJ2IyNMAnjZiX+Q6tCC3ZGrlzJkzMJlMcHPjvWxEGv42kMOcOXMGAGDuye+2bduisrISZ8+etWW3iJwOg5wcJj8/H6GhofDwMO+FIa8lJ2oYg5wcpqCgwOzROMAgJ2oMg5wcJj8/36Ig19pyaoWoLgY5OUxBQQEiIiLMbh8eHl6zHRFdxSAnh7F0RM4gJ2oYg5wcQkQsHpF7enoiKCiIUytE9TDIySHOnz+PiooKi0bkABAREcEROVE9DHJyCG0FTEtG5EDV9ApH5ER1McjJIbRRNUfkRPoxyMkhtBG5NUHOETlRXQxycghtVG3N1ApH5ER1McjJIfSMyC9evIiysjJbdIvIKTHIySEKCgrg4+MDPz8/i7bTriXn9ArRVQxycgjtZqCq9yIxnzYVw+kVoqsY5OQQBQUFNaNrS2hBzhE50VUMcnKIwsJCq4Kct+kTXYtBTg5hbZBzRE50LQY5OURhYSHCwsIs3o4jcqJrMcjJ7kTE6iD38vJCYGAgg5yoFgY52V1xcTEqKyutCnKA660Q1ccgJ7srLCwEAKuDnOutENXFICe7MyLIOSInuopBTnanN8i53gpRXYYEuVIqRCn1oVLqoFLqgFLqeiP2S60TR+RExvIwaD8vA9gkIhOUUl4ALFtAg1yKESPyCxcu4MqVK/Dy8jKya0ROSfeIXCkVDOBGAKsBQESuiEiR3v1S66UFeWhoqFXb86YgorqMmFrpDCAfwJtKqb1KqVVKKf/6jZRSs5RSaUqpNG0JU3JNhYWF8PPzg4+Pj1XbM8iJ6jIiyD0A9Afwqoj0A3AJwOP1G4nIChEZICIDLF2DmloXa28G0vDuTqK6jAjyHAA5IpJa/fWHqAp2ogbpDXKOyInq0h3kInIKwAmlVI/qh0YA+EXvfqn14oicyFhGXbUyB8A71VesHAUw06D9UitUWFiIHj16NN+wEQxyoroMCXIR+QHAACP2Ra2f3hG5j48PAgICOLVCVI13dpJd6Vn5sDbe3Ul0FYOc7Ory5csoKyuz6k0lamOQE13FICe70ntXpyY8PBznzp0zoktETo9BTnZlVJCHhYXV7IvI1THIya4Y5ETGY5CTXRk5tVJYWAiTyWREt4icGoOc7MrIEbnJZMKFCxeM6BaRU2OQk10ZGeS190fkyhjkZFeFhYXw9vaGr6+vrv0wyImuYpCTXWk3AymldO2HQU50FYOc7MqIuzoBBjlRbQxysisGOZHxGORkV0YFufY2cVw4i4hBTnZmVJB7eXkhMDCQI3IiMMjJzowKcoB3dxJpGORkN6Wlpbh8+TKDnMhgDHKyG221QgY5kbEY5GQ3Rt3VqWGQE1VhkJPdMMiJbINBTnajXSqo992BNFqQi4gh+yNyVgxyshujR+Th4eGoqKhAcXGxIfsjclYMcrIbW0yt1N4vkatikJPdFBYWwsPDAwEBAYbsj0FOVMWwIFdKuSul9iqlNhi1T2pdjFr5UMMgJ6pi5Ij8IQAHDNwftTJG3tUJMMiJNIYEuVIqBsBvAawyYn/UOjHIiWzDqBH5MgB/BNDoO+EqpWYppdKUUmn5+fkGlSVnYqsg5wqI5Op0B7lS6ncAzohIelPtRGSFiAwQkQGRkZF6y5ITMjrIvb294e/vzxE5uTwjRuQ3ABinlDoG4D0AKUqpfxqwX2pljA5ygHd3EgEGBLmI/ElEYkSkE4DJALaLyHTdPaNWpby8HMXFxQxyIhvgdeRkF0avfKhhkBMZHOQi8qWI/M7IfVLrYPRdnRoGORFH5GQnDHIi22GQk13YMsjPnj3LFRDJpTHIyS60a72NDvLw8HCUl5fj0qVLhu6XyJkwyMkutBG5UWuRa3h3JxGDnOyksLAQbm5uCAoKMnS/DHIiBjnZydmzZxEaGgo3N2N/5BjkRAxyspPCwkLDp1UABjkRwCAnOzl79qzhJzoBBjkRwCAnO7H1iJwrIJIrY5CTXdhiwSwA8PX1ha+vL0fk5NIY5GQXZ8+etcmIHODdnUQMcrI5W618qGGQk6tjkJPN2er2fA2DnFwdg5xszlZ3dWoY5OTqGORkc7ZaZ0XDICdXxyAnm7PHiJwrIJIrY5CTzdl6jjw8PBxlZWUoKSmxyf6JWjoGOdmcPaZWAN7dSa6LQU42V1hYCHd3d8NXPtQwyMnVMcjJ5rR1VpRSNtk/g5xcHYOcbM5W66xoGOTk6hjkZHO2WmdFo+27oKDAZjWIWjIGOdmcrZaw1Wijfa6ASK5Kd5ArpToopXYopX5RSu1XSj1kRMeo9bD11Iqfnx/8/PwY5OSyPAzYRwWAeSKSoZQKBJCulNoiIr8YsG9qBWw9IgeqRuWcWiFXpXtELiJ5IpJR/XkxgAMA2uvdL7UOZWVluHTpkk1H5AAQERHBETm5LEPnyJVSnQD0A5DawHOzlFJpSqm0/Px8I8tSC2bruzo1HJGTKzMsyJVSAQD+DeBhEblQ/3kRWSEiA0RkQGRkpFFlqYWzV5BHREQwyMllGRLkSilPVIX4OyLykRH7pNbB1gtmaTi1Qq7MiKtWFIDVAA6IyEv6u0Stia3XWdGEh4fj3LlzqKiosGkdopbIiBH5DQDuBJCilPqh+uNmA/ZLrYA9R+QAcO7cOZvWIWqJdF9+KCJfA7DNIhrk9Ow5Igeq7u7kORhyNbyzk2yqsLAQnp6eCAgIsGkdbUTOE57kihjkZFPaOiu2WvlQowU5T3iSK2KQuzhbvz2aPe7qBOpOrRC5Gga5ixIRLFq0CH5+fhgyZAh++OEHm9Sx9TorGo7IyZUxyF3U888/jyeffBLJycnIzMzEpEmTUFpaangde43I/fz84OPjwxE5uSQGuQs6ffo0Fi1ahFtvvRWff/451q5di8OHD2PRokWG18rPz7fbVSQRERGw1fIPJpMJ+/btw+7du22yfyI9GOQuaNGiRSgpKcHixYuhlMJNN92E2267Da+++irKy8sNqyMidr0csE2bNjYJcpPJhPHjx6NPnz4YPHgwZs6ciStXrhheh8haDHIXU1paijVr1mDy5Mno3r17zeO///3vUVBQgM8//9ywWhcuXEB5eXnN/LWtRUZG2iTIFy1ahE8++QRPPfUUnnjiCbz11ltYvHix4XWIrMUgdzHr169HUVER7r777jqPjxo1Cm3atMGaNWsMq6WFqj1H5GfOnDF0n2fOnMGCBQswadIkPPvss1i4cCEmTpyIRYsWISsry9BaRNZikLuYNWvWICYmBsOHD6/zuKenJyZNmoQNGzagpKTEkFqOCHKjR+QrV67ElStX8Mwzz9RcC7906VIopbBw4UJDaxFZi0HuQi5cuIAvvvgCU6ZMgbu7+zXPjx49GmVlZfj2228NqaddQWKvII+MjMTly5dx6dIlQ/ZXUVGBV199FSNHjkTPnj1rHo+JicH06dOxbt06ru1CLQKD3MGys7Pxt7/9DatWrcLFixdtWmvz5s2oqKjAuHHjGnz+xhtvhIeHB7Zu3WpIPW10bK858jZt2gCAYdMru3btQm5uLu67775rnps9ezZKSkrw9ttvG1KLSA8GuQNt2rQJPXr0wKOPPop7770XvXr1wvHjx21Wb8OGDQgLC0NSUlKDzwcEBCApKQnbtm0zpJ69p1a0OkZNr2zYsAFeXl4YPXr0Nc/17dsXSUlJeOONNwypRaQHg9xBMjIycOuttyI+Ph5HjhzBjh07UFxcjNGjRxs2NVBbZWUlNm7ciDFjxsDDo/FFL0eMGIG0tDRDpgwKCgrg6+sLf39/3fsyh5EjchHBp59+ipSUlEYX/JoyZQp+/vlnHDx4UHc9Ij1cPshFBC+//DLi4+MRFhaGe++9F6dPn7ZpzcrKSsyaNQuhoaHYsmULunTpgmHDhuHDDz/EgQMHsGTJEsNr7t27F/n5+bj55qaXik9OToaIIDX1mrddtVh+fr7dplWAq0FuxIj80KFDyMzMxNixYxttM2HCBCil8K9//Ut3vab8+OOPWLx4MVasWGH4VTnUOrh0kIsIZsyYgYcffhgREREYM2YM1q5di+TkZJw8edJmdVetWoX09HQsW7aszjokI0eOxMSJE/HCCy8gNzfX0Jrbt28HAKSkpDTZLjExEUopQ4Lc3muDa7WMCLvNmzcDQJN/+KKjozF06FCbBbmIYMGCBejbty8ef/xx3HfffejZs6dhU1/Uerh0kC9duhRr167F/PnzsXPnTrzzzjvYunUrcnNzcccdd6CystLwmuXl5fjrX/+KpKQkTJw48Zrnn3/+eVy5cgXLli0ztO727dsRHx+PqKioJtsFBQUhPj4e3333ne6a9rw9HwD8/f3h5+dnyIj8q6++QmxsLDp16tRku/Hjx2P//v3IzMzUXbO+JUuWYP78+Zg+fTpOnz6NvXv3on379hgzZgyXCqC6RMTuH4mJieJo+/fvFw8PD7n99tvFZDLVeW7t2rUCQF588UXD665Zs0YAyKefftpomzvuuENCQ0Pl0qVLhtQsKysTf39/efDBB81qP3PmTAkPD7/m/8VSnTt3lmnTplm83cTXvpWJr31rVc1OnTrJnXfeadW2GpPJJG3btjWr70eOHBEA8tJLL+mqWd/333/f4M9nYWGhdOnSRdq3by/5+fmG1qzvypUrsn37dlm+fLl88cUXcuXKFZvWo+YBSJMGMtUlg9xkMslNN90kISEhDf4ymEwmGTt2rAQEBMiZM2cMrZ2YmCi9e/duMiR37twpAGTlypWG1Pz6668FgPz73/82q/1rr70mAOTIkSO66gYGBsrDDz9s8XZ6gnzgwIEyevRoq7bVHD58WADIa6+9Zlb73r17y/Dhw3XVrM1kMsmgQYMkOjpaCgsLr3l+79694uHhIffcc49hNev7/vvvpXfv3gKg5iMuLk527Nhhs5r1VVRUSEVFhd3qOYPGgtwlp1a2bduGLVu24JlnnmnwZJxSCosXL8bly5cNXVPjxx9/RHp6Ou67774m3zFn6NCh6N27N958801D6u7YsQNKKSQnJ5vVftCgQQCga3qltLQUxcXFdn//zDZt2ug+Wf3VV18BqPo+mGPcuHHYtWuXYTcHffTRR/juu+/w3HPPITQ09Jrn+/bti0cffRRvvPGGTaZYvvrqKyQnJ+P8+fN49913kZ2djY8++ghKKYwaNcrmJ3czMjIwbtw4hISEwMfHB3379sWqVatsMtVZX2VlJVJTU7Fy5Uq8//77yMvLs3lNQzSU7rb+cPSIfNSoUdKuXTspLS1tst2MGTPEx8fHsFH53LlzxcvLS86ePdts2+eff96QUbGIyPDhw6Vv375mty8rKxMPDw95/PHHra557NgxASCrV6+2eFs9I/J77rlH2rVrZ9W2mv/+7/+WsLAws6eWvv32WwEg69at01VXRKSyslLi4+MlPj6+ydFocXGxREVFSXJysu6atR0+fFgCAwOlR48ekpeXV+e5c+fOyZAhQ8TDw8MmI3OTySSLFy8WpZRERETIgw8+KH/605+kf//+AkBSUlIMf4Vc2+bNm695FeLu7i533nmnzaexzAVOrVT58ccfBYAsWrSo2bb79+8XAPLcc8/prltaWiphYWEyadIks9pnZ2cLAHn22Wd11S0pKRFvb2955JFHLNquT58+8tvf/tbqunv27BEAsmHDBou31RPkTz75pLi7u+t6Sd6/f38ZOXKk2e0rKiokMjJSpkyZYnVNzWeffSYAZO3atc22ffnllwWAbN++XXddkaqfleuuu07CwsIkOzu7wTbnzp2Tnj17Snh4uOTm5hpSV/PYY48JAJk4caIUFRXVPG4ymWT16tXi4+MjvXv3llOnThlaV0Rk4cKFopSS7t27y9tvvy1Hjx6VjIwMmTdvnnh6ekq7du3khx9+MLyupRjk1e666y7x9/c3a1QsInLTTTdJdHS07hM977//vgCQzZs3m71NcnKy9OzZU9dJx+3btzd7crUh06ZNkw4dOlhd95NPPhEAkpaWZvG2eoL8lVdeEQBW/7KXlZWJp6en/PGPf7Rou5kzZ0pISIjun5OUlBRp3769WfspKSmR6OhoGTp0qO4T0yIizz77rFk/KwcPHhRfX18ZM2aMIXVFrv5Ruv/++6WysrLBNtu3bxc/Pz9JTEyUy5cvG1JX5OrPzNSpUxu8wGDv3r0SExMjQUFB8v333xtW1xo2DXIAowEcApAJ4PHm2jsqyHNycsTT01PmzJlj9jaffvqpAJD33ntPV+1Ro0ZJbGysRSPF//u//xMA8vPPP1td96mnnhI3N7c6IxxzaFM7DZ1sM8frr78uAOTEiRMWb6snyP/9738LANm7d69V22dkZFj1/f7oo490j45/+eUXASB//etfzd7mH//4hwCQrVu3Wl1XRCQrK0t8fHzkjjvuMKv93//+d8NOyO/evVs8PDzklltuafb3Y/369aKUkqlTpxryR+TTTz8VNzc3GTt2bJO1jx8/Lh07dpQ2bdoYMt1pLZsFOQB3AEcAdAHgBeBHAPFNbeOoIH/sscfEzc3Nom9EZWWldO3aVQYPHmx13ezsbFFKyfz58y3a7tSpU+Lm5mbxdrVdf/31MmjQIIu327hxowCQXbt2WVVXG92VlZVZvK2eINfmqzdu3GjV9qtXrxYAcujQIYu2Ky4uFi8vL4unsGp75JFHxNPTU06fPm32NiUlJdK+fXu54YYbdAXb+PHjxc/Pr9EplfoqKyslJSVFAgIC5OjRo1bXLSwslI4dO0qnTp3k3LlzZm2zcOFCASCLFy+2uq6ISHp6uvj7+0v//v2luLi42fYHDhyQsLAwiYuLs+lcfVNsGeTXA/ii1td/AvCnpraxNsh3794t//jHP6za9sKFCxIcHGz2iKO2ZcuWCQCrX1ZpoZaVlWXxtsOGDZNevXpZVff8+fPi7u4uTzzxhMXb5uTkCAB55ZVXrKo9e/ZsCQsLs2pbPUF+9OhRq0+yiog8+OCDEhAQ0OjL+6aMGTNGunbtalWglpSUSFhYmFU/n8uXL9c1Kv/iiy8EgCxcuNCi7bKzsyUoKEiGDx9u1f+XyWSS8ePHi4eHh6Smplq03aRJk0QpJV988YXFdUWqRtjt2rWTDh06yMmTJ83e7uuvvxYfHx8ZPHiwlJSUWFVbD1sG+QQAq2p9fSeAVxpoNwtAGoC02NhYqw5i7ty54uHhYdXLdS2M9+zZY/G2RUVFEhAQIHfddZfF21ZWVkqnTp1kxIgRFm8rcvWXdN++fRZvq00Lbdu2zeJtTSaThIWFyaxZsyzeVqRqhBcfH2/VtnqCvKSkRNcJ6sGDB8uQIUOs2labCvvll18s3nbdunUWn0PRlJaWSvv27WXIkCEW/xEpKyuTHj16SLdu3Zq9iqshK1euFADy6quvWrytNjdtzY13Fy9elISEBAkPD5djx45ZtO358+elT58+EhQUZNW05QcffCAAZPr06Rb/f1vyR6MhDg/y2h/WjsizsrLE3d1d5s2bZ9F25eXl0rFjR6t/QUVEHnjgAfH29rb4MqRt27bpujQtLy9PlFLy9NNPW7ztQw89JD4+PlaPHJKTkyUpKcmqbW+44QZJSUmxals9QS4iEhISYvZdrLVVVFSIn5+fzJ0716q6x48ft/ol/7Bhw6Rz585WjWxFrv7B37Jli0XbvfDCCwJAPvvsM6vqajfXBQQEWPSKc+/eveLl5SU333yz1cd8+PBhCQoKkgEDBpj9M37lyhUZNWqUeHh4WPVHU7NgwQKLX8Vs2rRJfHx8LL7woLZWMbUiIjJ58mQJDAy06OTde++9JwDk448/trruvn37BIA8//zzFm03depUCQkJ0XWW/cYbb5TevXtbvF1CQoJFl9HVN2fOHKunGbp27Wr15Xh6g7xXr15y++23W7zdgQMHBIC89dZbVtfu16+f3HDDDRZtc+jQIbMviW2MNiq3ZK48JydHAgICZOzYsVbXFamaYgkMDJSUlBSzflYuXLggcXFxEh0drfv67I8//lgAyLRp05o9bpPJJPfee68AkFWrVumqazKZZNq0aQJA3njjjWbbp6amir+/v1x33XVy/vx5q+vaMsg9ABwF0LnWyc7eTW2jJ8jT09MtGvWYTCYZOHCgxMXFWf2XXzNs2DDp2LGj2VeeFBYWire3t1Wjw9q0KxMsecmel5dn8RUQ9Wkvm605S+/v72/1iT+9QZ6SkmLVyel33nlHAMiPP/5ode358+eLm5ubRQE1b948cXd31/2yWxuVb9q0yaz2U6dOFW9vb8nMzNRVV0Rk1apVZt33UFlZKRMmTBA3NzfZuXOn7roiIs8995wAkEceeaTRMDeZTPLoo48KAHnyyScNqVtaWiqjRo0SNzc3WbFiRaPtvvrqKwkKCpLOnTu33KmVqn3jZgCHq69eebK59nqvWhkxYoRER0ebdUXE1q1brZ7Dq+/DDz8UAPLJJ5+Y1V77xUpPT9dV9+TJk6KUkr/85S9mb6OFkp7rXrWbev7zn/9YtF1xcbFVr140eoN86tSp0rlzZ4u3mzdvnnh7e+u6Fvz7778XAPL222+b1f7y5csSGhpq1UnO+kpLS6VLly4SHx/f7DFs3rxZAMhTTz2lu65IVVDeeeedopSSf/3rX422+d///V8BIEuXLjWkrrbfOXPmCACZOXPmNdeCX7p0SaZPny4AZM6cOYZd+y5SNVc/evRoASCzZs2qM1NQWloqS5YsEU9PT+nevbtV5/bqs2mQW/qhN8g3bdokAOT1119vsp3JZJKhQ4dKdHS0IWeYy8vLJSYmxqzpCpPJJL169ZL+/fvrrisiMnToUElISDC7/T333CMhISG67nAsLi4WpZTFd5dmZmbqmqLQG+Tz5s0THx8fi39hU1JSZMCAAVbXFakacUZHR5s9tfPGG28IAMNuedduxFqyZEmjbc6fPy+xsbHSo0cPQ2+suXTpkgwZMkQ8PT3ln//8Z53nysrKasJ29uzZhoapSNXv25///GcBIF26dJEXX3xRNmzYIC+88ILExMTUvFowuq5I1bz7vHnzxM3NTfz9/WXcuHFyxx13SFRUlACQW265xer7MeprVUFuMpnk+uuvl3bt2snFixcbbaeNOqy9ZLEhixYtMuuORe2PzZo1awypq92AceDAgWbbmkwm6dixo9x2222663br1k0mTJhg0Tbaaouff/65VTX1BvnSpUsFgNl374pU/Z+FhIRYfZVObbNnzxZfX99mz+OYTCazVsO0hMlkknHjxomXl5dkZGRc83xlZaXcdttt4ubmJt9+a/3/cWOKiopk6NChAkDGjh0rb775prz88suSkJDQ7PSHEXbs2CGDBg0S1FovZdCgQfLll1/arKYmIyND7r33XomPj5cuXbrIhAkTZPPmzYYeb6sKcpGqeScAjV7NUVZWJr169ZLOnTsber1nUVGRhISEyK233tpku5EjR0pUVJRVN8Q0JDc31+zRsXZi1ojppNtuu026d+9u0TbacgTW3pGqN8i1y8MsWRsjKytLYMHStU1JTU0VAE3Om9Zut3z5ct01a8vPz5fo6Gjp2LFjnfMblZWV8tBDDwlg/PrptZWXl8uzzz4rERERNWEaHx8v69evt1nN+o4dOya7d+82fD0YR2t1QS5SdQWLp6dng7djazfhWLNoU3Pmz58vABq9iUG75LCpl7fWGDJkiPTp06fZdgsWLBCllO4TKyJXT95Z8hJcGxGbe6defXqDXAtIS4JDu7XfkhtTGmMymSQ+Pr7ZE6533XWXBAQE6LqKoTFpaWkSFhYmbdq0kZdeeknee+89GTFihACQuXPn2nRUrCkrK5PDhw9bfJ03Na5VBnlBQYFERUVJt27d6pxI+Pjjj2vWY7CF8+fPS7t27SQxMfGaOeiKigpJTEyU2NhYw+/80hYWOnjwYJPt+vXrJ9dff70hNbXRrSUnbB9++GHx9/e3Oiz0Brl2xY4ld6U+9dRT4u7ubtic8YsvvtjkFFxWVpZ4enrK//zP/xhSryH79u2TIUOG1IyKIyMjZfny5XYJcbKNVhnkIlVrawQGBkpMTIwsXrxY5syZI25ubpKYmGjYW6U15N13321wakc74fLuu+8aXlO7bX7BggWNttGmCF544QVDamrXVpt7FYaIyIQJE6RHjx5W19Qb5JWVleLl5WXRCoY333yzRSeTm1NUVCTBwcEyfvz4Bp+/++67xdvbW3Jycgyr2RCTySSHDh2S9PR0h9xSTsZqtUEuUnXJl3aCQyklM2fOtMnL1dpMJpPMmDFDAMgzzzwjOTk5NWsa2/ItuIYOHSpdu3Zt9GqUp59+WpRSuhYyqq28vFy8vb3lD3/4g9nbJCUl6boRSW+Qi1TdkDR58mSz20dFRVm1BENTtD/q9V/N7NmzR9zc3OTRRx81tB61fq06yDU5OTmGXk7VnLKyMpk4cWKdM+STJ0+2aR+0E4kNzf9ql0f+13/9l6E1+/bta9H7YMbExMjdd99tdT0jgnz48OFm3xR08uRJASDLli3TVbO+wsJCiY6Olp49e9a8OiwqKpKuXbtKbGys1ecQyHU1FuSt6j0727dvD19fX7vV8/Lywvvvv49du3bhlVdewZYtW7Bu3Tqb9mH8+PHo0KEDli5dWvWXuJaNGzciJycHs2bNMrRmnz598PPPP5vVtqKiAidPnkRMTIyhfbBUbGwsjh8/blbbvXv3AgD69etnaB9CQ0OxZs0aHDp0CMnJyXjttdeQlJSEY8eO4Z///CdCQkIMrUeuq1UFuaMMHToUDz74IEaOHNnkmyobwcPDA3/4wx+wc+dOfPLJJzWPV1RU4IknnkCnTp0wduxYQ2smJCQgNzfXrDcXPnXqFEwmk8ODvGPHjjh58iTKy8ubbasFed++fQ3vx4gRI/DBBx/g2LFjeOCBB1BUVIStW7ea/cbOROZgkDuh2bNno0+fPpg7d27NO8a/9NJL2L9/P1566SV4enoaWi8hIQEAsG/fvmbb5uTkAAA6dOhgaB8sFRsbC5PJhJMnTzbbNiMjA3FxcQgKCrJJX26//XZkZ2fj+PHjOH78OIYNG2aTOuS6GOROyMPDAytXrkRBQQGSkpIwZcoUPPbYYxg3bhxuvfVWw+v16dMHAMyaXjlx4gQAOHxEHhsbCwDIzs5utm1GRobh0yr1+fn5oUOHDob/kSUCGORO6ze/+Q2+/PJL+Pn5YfPmzbj//vvxwQcf2GRqJyYmBsHBwWaNyFtKkHft2hUA8OuvvzbZ7ty5czh27Bj69+9vj24R2YSHoztA1hs0aBD2799v8zpKKSQkJJg1Ij969CiCg4MRGhpq8341pWPHjvD29sahQ4eabGerE51E9sQROZklISEB+/btu+ZKmfoyMzPRrVs3m5/0bY67uzu6devWbJBnZGQAYJCTc2OQk1n69OmDoqIi5ObmNtnu119/Rbdu3ezUq6b16NGj2SBPTU1Fp06dEBkZaadeERmPQU5m0U54/vTTT422uXLlCo4dO4a4uDh7datJ3bt3x5EjR5q8BDE1NRVJSUl27BWR8RjkZBbtGuv09PRG22RnZ8NkMrWoEXlFRQWysrIafD4vLw8nTpzAb37zGzv3jMhYDHIyS1BQELp3795kkGtXiLSkIAfQ6PRKamoqADDIyekxyMlsAwYMQFpaWqPPZ2ZmAkCLmVrRgvzgwYMNPp+amgpPT0+e6CSnxyAnsyUmJiI3N7fmbtL6MjMzERgY2GJOHIaFhSE2NrbRVxFff/01+vXrBx8fHzv3jMhYDHIy24ABAwA0Pk9+4MABxMXFOfzSw9qSkpKwZ8+eax4vLi7Gnj17MGLECAf0ishYDHIyW79+/aCUwnfffXfNcyKC9PR0JCYmOqBnjUtKSkJ2djby8vLqPL5z505UVFRg5MiRDuoZkXEY5GS2wMBA9OvXDzt37rzmuaysLJw7d65m1N5SaJcWaic2NVu3boWPjw8GDx7siG4RGUpXkCulliilDiqlflJK/UcpFWJQv6iFGjZsGHbv3o3S0tI6j2snQVvaiLxfv37w9PTE7t27ax4TEWzevBk33ngj58epVdA7It8CIEFE/h+AwwD+pL9L1JINGzYMZWVl18w7p6enw8vLq2bJ25bCx8cHAwcOxKZNm2qWF0hPT8eBAwdsslIkkSPoCnIR2SwiFdVf7gHg2CXvyOaGDh0KNzc3fPnll3UeT0tLQ58+feDt7e2YjjXhzjvvxE8//VSzrsqqVavg6+uLqVOnOrhnRMYwco78HgCfG7g/aoFCQkKQmJiIzz77rOaxkpISfPfddxg0aJADe9a4KVOmwNfXF6tWrcKZM2ewbt06TJw4EcHBwY7uGpEhmg1ypdRWpdS+Bj5uqdXmSQAVAN5pYj+zlFJpSqm0/Px8Y3pPDjF16lSkpaXVLKG7YcMGXLx4EbfffruDe9aw4OBgTJ48GStXrsTAgQNRUVGBRx55xNHdIjJMs0EuIiNFJKGBj08AQCl1N4DfAZgmTaxxKiIrRGSAiAxoKTeMkHWmTZsGDw8PvPXWWwCAdevWISoqqkW/hdnf/vY3zJgxA2fPnsV//vMfXHfddY7uEpFhdL2xhFJqNIA/AkgWkcvGdIlausjISPzud7/D6tWrMXDgQGzcuBGzZ8+Gu7u7o7vWqODgYKxevRorVqxo0f0ksobeOfJXAAQC2KKU+kEp9ZoBfSInsGTJEnh7e2PSpElo06YN5syZ4+gumYUhTq2RrhG5iLSMZe7I7rp164bt27fj9ddfx+OPP46oqChHd4nIZfE9O8lqvXr1wrJlyxzdDSKXx1v0iYicHIOciMjJMciJiJwcg5yIyMkxyImInByDnIjIyTHIiYicHIOciMjJqSbWubJdUaXyAWRbuXkEgAIDu+MMeMyuwxWPm8dsvo4ics2qgw4Jcj2UUmki0rLeGNLGeMyuwxWPm8esH6dWiIicHIOciMjJOWOQr3B0BxyAx+w6XPG4ecw6Od0cORER1eWMI3IiIqqFQU5E5ORabJArpUYrpQ4ppTKVUo838Ly3Uur96udTlVKdHNBNQ5lxzI8qpX5RSv2klNqmlOroiH4aqbljrtXudqWUKKWc/jI1c45ZKTWx+nu9Xym1zt59NJoZP9uxSqkdSqm91T/fNzuin0ZSSr2hlDqjlNrXyPNKKfX36v+Tn5RS/a0uJiIt7gOAO4AjALoA8ALwI4D4em1mA3it+vPJAN53dL/tcMzDAfhVf/6AKxxzdbtAALsA7AEwwNH9tsP3OQ7AXgCh1V+3cXS/7XDMKwA8UP15PIBjju63Acd9I4D+APY18vzNAD4HoAAkAUi1tlZLHZEPApApIkdF5AqA9wDcUq/NLQDerv78QwAjlFLKjn00WrPHLCI7RORy9Zd7AMTYuY9GM+f7DAALACwGUGrPztmIOcd8L4DlInIOAETkjJ37aDRzjlkABFV/HgzgpB37ZxMisgtAYRNNbgGwRqrsARCilGpnTa2WGuTtAZyo9XVO9WMNthGRCgDnAYTbpXe2Yc4x1/Z7VP01d2bNHnP1y80OIvKZPTtmQ+Z8n7sD6K6U+kYptUcpNdpuvbMNc475GQDTlVI5ADYCmGOfrjmUpb/zjeKbLzshpdR0AAMAJDu6L7aklHID8BKAux3cFXvzQNX0yjBUverapZTqIyJFjuyUjU0B8JaILFVKXQ9grVIqQURMju6YM2ipI/JcAB1qfR1T/ViDbZRSHqh6OXbWLr2zDXOOGUqpkQCeBDBORMrs1Ddbae6YAwEkAPhSKXUMVfOI6538hKc53+ccAOtFpFxEsgAcRlWwOytzjvn3AP4FACKyG4APqhaWas3M+p03R0sN8u8BxCmlOiulvFB1MnN9vTbrAcyo/nwCgO1SfQbBSTV7zEqpfgBeR1WIO/u8KdDMMYvIeRGJEJFOItIJVecFxolImmO6awhzfrY/RtVoHEqpCFRNtRy1Yx+NZs4xHwcwAgCUUr1QFeT5du2l/a0HcFf11StJAM6LSJ5Ve3L0md0mzvjejKqRyBEAT1Y/9iyqfpGBqm/0BwAyAXwHoIuj+2yHY94K4DSAH6o/1ju6z7Y+5nptv4STX7Vi5vdZoWpK6RcAPwOY7Og+2+GY4wF8g6orWn4AMMrRfTbgmN8FkAegHFWvsn4P4H4A99f6Pi+v/j/5Wc/PNm/RJyJyci11aoWIiMzEICcicnIMciIiJ8cgJyJycgxyIiInxyAnInJyDHIiIif3/wFwBwbUJCpJsgAAAABJRU5ErkJggg==\%0A}
